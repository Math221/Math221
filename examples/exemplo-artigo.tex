%% exemplo-artigo.tex
%% Exemplo de artigo usando a classe Lumina
%% Autor: Matheus Rodrigues Soares
%% Data: 2024-12-28

\documentclass[
    fonte=Arial,           % Fonte Arial
    layout=article,        % Layout para artigo
    nbr6024=2012          % Versão da norma NBR 6024
]{lumina}

%% Pacotes adicionais
\usepackage[brazil]{babel}
\usepackage{lipsum}
\usepackage{amsmath}
\usepackage{graphicx}

%% Dados do artigo
\title{Aplicação da Classe Lumina em Artigos Científicos: Uma Abordagem Prática}
\author{Matheus Rodrigues Soares\thanks{Universidade Federal do Exemplo, Programa de Pós-Graduação em Ciências. E-mail: matheussoares.ivp@gmail.com}}
\date{Dezembro de 2024}

\begin{document}

\maketitle

\begin{abstract}
Este artigo apresenta a aplicação da classe Lumina para a formatação de artigos científicos em conformidade com as normas ABNT. A classe oferece flexibilidade e facilidade de uso, permitindo que pesquisadores foquem no conteúdo científico. Os resultados demonstram a eficácia da ferramenta na produção de documentos acadêmicos de alta qualidade.

\textbf{Palavras-chave:} LaTeX. Formatação acadêmica. ABNT. Lumina.
\end{abstract}

\section{Introdução}

\lipsum[1]

A formatação de trabalhos acadêmicos segundo as normas da ABNT representa um desafio constante para pesquisadores e estudantes. A classe Lumina foi desenvolvida para simplificar esse processo, oferecendo uma solução moderna e flexível.

\section{Metodologia}

\lipsum[2-3]

\subsection{Desenvolvimento da Classe}

O desenvolvimento da classe Lumina seguiu os seguintes princípios:

\begin{itemize}
    \item Conformidade estrita com as normas ABNT
    \item Flexibilidade de configuração
    \item Facilidade de uso
    \item Suporte a tecnologias modernas (XeLaTeX, Unicode)
\end{itemize}

\section{Resultados}

\lipsum[4]

A Equação \ref{eq:exemplo} ilustra a qualidade tipográfica alcançada:

\begin{equation}
E = mc^2
\label{eq:exemplo}
\end{equation}

\subsection{Características Principais}

\lipsum[5-6]

\section{Discussão}

\lipsum[7-8]

\section{Considerações Finais}

A classe Lumina representa um avanço significativo na produção de trabalhos acadêmicos brasileiros, oferecendo uma solução robusta e flexível para a comunidade científica.

\section*{Referências}

LAMPORT, L. \textbf{LaTeX}: A Document Preparation System. 2. ed. Boston: Addison-Wesley, 1994.

MITTELBACH, F. et al. \textbf{The LaTeX Companion}. 2. ed. Boston: Addison-Wesley, 2004.

\end{document}