%% exemplo-basico.tex
%% Exemplo básico de uso da classe Lumina
%% Autor: Matheus Rodrigues Soares
%% Data: 2024-12-28

\documentclass[
    fonte=TimesNewRoman,    % Fonte padrão
    nbr6024=2012,          % Versão da norma NBR 6024
    layout=thesis,         % Layout para tese/dissertação
    estilocaso=all         % Capitalização de títulos
]{lumina}

%% Pacotes adicionais (opcional)
\usepackage[brazil]{babel}
\usepackage{lipsum}        % Para texto de exemplo

%% Configuração dos dados do trabalho
\title{Exemplo de Trabalho Acadêmico com a Classe Lumina}
\author{Nome do Autor}
\date{2024}

\instituicao{Universidade Federal do Exemplo}
\programa{Programa de Pós-Graduação em Ciências}
\orientador{Prof. Dr. Nome do Orientador}
\local{Cidade}

\begin{document}

%% Elementos pré-textuais
\capa
\folhaderosto

%% Sumário
\tableofcontents
\clearpage

%% Elementos textuais
\chapter{Introdução}

\lipsum[1-2]

Este é um exemplo básico de uso da classe Lumina para trabalhos acadêmicos em conformidade com as normas ABNT.

\section{Objetivos}

\lipsum[3]

\subsection{Objetivo Geral}

\lipsum[4]

\subsection{Objetivos Específicos}

\lipsum[5]

\chapter{Metodologia}

\lipsum[6-8]

\section{Materiais e Métodos}

\lipsum[9-10]

\chapter{Resultados e Discussão}

\lipsum[11-13]

\section{Análise dos Dados}

\lipsum[14-15]

\chapter{Considerações Finais}

\lipsum[16-17]

%% Elementos pós-textuais
\chapter*{Referências}
\addcontentsline{toc}{chapter}{Referências}

[1] EXEMPLO, A. B. \textbf{Título do trabalho}: subtítulo. Cidade: Editora, 2024.

[2] OUTRO, C. D. Artigo de exemplo. \textbf{Revista Científica}, v. 1, n. 1, p. 1-10, 2024.

\end{document}